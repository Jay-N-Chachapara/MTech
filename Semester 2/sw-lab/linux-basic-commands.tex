\documentclass{article}
\title{Linux Basic and Advanced Commnds}
\author{Jay Chachapara}
\date{\today}
\begin{document}
	\maketitle
	\begin{itemize}
		\item \textbf{File Commands}
			\begin{itemize}
				\item \textbf{ls} \\					
					dirictory listing \\
					list directory content of files and directories.\\
					-t sorts file by modification time (last modified is first)\\
					-1 one file per line \\
					-a display all files including hidden files \\
					can merge/call more than one command in one \\ i.e. ls -t -1\\
				\item \textbf{ls -al} \\
					display all files including hidden file + details \\
				\item \textbf{cd dir} \\
					Change directory to dir\\
					cd .. one level up dir\\
				\item cd \\
					The cd command will allow you to change directories. When you open a terminal you will be in your home directory. To move around the file system you will use cd.\\
				\item pwd \\
					stads for Print Working Directory\\
					shows current working directory\\
					-L (Logical) use PWD from environment, even if it contains symbolic links\\
					-P (Physical) Avoid all symbolic links\\
					If both ‘-L‘ and ‘-P‘ options are used, option ‘L‘ is taken into priority. If no option is specified at the prompt, pwd will avoid all symlinks, i.e., take option ‘-P‘ into account.\\
					Exit status 0 for sucess and non zero for failure\\
				\item \textbf{mkdir dir}\\
					create a dirictory dir\\
					stands for make directory\\
					mkdir \{dir1,dir2,dir3\} - creates multiple directories in the current location. Do not use space inside \{\}. \\
					mkdir -p directory/path/newdir - creates a directory structure with the missing parent directory (if any)\\
					mkdir -m777 dir - creates a directory and sets full read, write, and execute permission for all users \\
				\item rm file 
				\item rm -r dir
				\item rm -f file
				\item rm -rf dir
				\item cp file1 file2
				\item cp -r dir1 dir2
				\item mv file1 file2
				\item ln -s file link
				\item touch file
				\item cat {$>$} file
				\item more file
				\item head file
				\item tail file
				\item tail -f file 
			\end{itemize}
		\item Process Management
			\begin{itemize}
				\item ps 
				\item top
				\item kill pid
				\item killall proc
				\item bg
				\item fg
				\item fg n
			\end{itemize}
		\item System Information
			\begin{itemize}
				\item date
				\item cal
				\item uptime
				\item w
				\item whoami
				\item finger user
				\item uname -a
				\item cat /proc/cpuinfo
				\item cat /proc/meminfo
				\item man command
				\item df
				\item du
				\item free
				\item whereis app
				\item which app
			\end{itemize}
		\item File Permissions
			\begin{itemize}
				\item chmod octal file
				\item chmod 777
				\item chmod 755
			\end{itemize}
	\end{itemize}
\end{document}